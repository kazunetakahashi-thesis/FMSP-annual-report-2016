%#!platex FMSP_AnnualReport2016.tex
%#BIBTEX pbibtex FMSP_AnnualReport2016

%%%%%%%%%%%%%%%%%%%%%%%%%%%%%%%%%%%%%%%%%%%%%%%%%%%%%%%%%%%%%%%%%%%%%%%%%%%% 
% 氏名(ローマ字綴りで名字は全て大文字,名前は最初の字だけ大文字) 
% を書いて下さい. {\bf 数理 太郎 (SURI Taro)}
%%%%%%%%%%% 
%%%%%%%%%%%%%%%%%%%%%%%%%%%%%%%%%%%%%%%

{\bf 高橋 和音 (TAKAHASHI Kazune)}

%学振DC1,国費などの留学生などに採用されている人は記載して下さい.
%(例) 学振DC1
%%%%%%%%%%%%%%%%%%%%%%%%%%%%%%%%%%%%%%%%%%%%%%%%%%%%%%%%%%%%%%%%

%%%%%%%%%%%%%%%%%%%%%%%%%%%%%%%%%%%%%%%%%%%%%%%%%%%%%%%%%%%%%%%%%%%%%%%%%%% 
% 所属専攻名と学年を記入して下さい
%%%%%%%%%%%%%%%%%%%%%%%%%%%%%%%%%%%%%%%%%%%%%%%
%(例) 数理科学専攻 修士課程1年
%物理学専攻 博士課程1年
%%%%%%%%%%%%%%%%%%%%%%%%%%% 

数理科学専攻 博士課程 2 年


\vspace{0.2cm}
\noindent
{\bf 研究概要}

\vspace{0.1cm}
%%%%%%%%%%%%%%%%%%%%%%%%%%%%%%%%%%%%%%%%%%%%%%%%%%%%%%%%%%%%%%%%%%%%%%%%%% 
%研究の要約を記入してして下さい.
%留学生の人などは英文でも結構です.
%コンパイルして0.5ページ以上2ページ以内程度になるようにまとめて下さい。
%%%%%%%%%%%%%%%%%%%%%%%%%%%%%%%%%%%%%%%%%%%%%%%%%%%%%%%%%%%%%%%%%%%%%%%%%% 

\begin{enumerate}
 \item[] \underline{Differential equations}
 \item {\bf Stand wave solutions of
       nonlinear Schr\"{o}dinger-Poisson systems~[5]} \\
       This is a joint work with Hiroyuki Miyahara (UTokyo).
       We worked on stand wave solutions
       of the following nonhomogeneous
       nonlinear Schr\"{o}dinger-Poisson systems:
       $-\Delta u -a \phi \left\lvert u \right\rvert^{q-1} u = \lambda u
       + b \left\lvert u \right\rvert^{p-1} u, -\Delta \phi =
       \left\lvert u \right\rvert^{q+1}$.
       There were many previous studies for the case the dimension $N = 3$,
       but our study covered the case $N \geq 3$.
       We proved existence and nonexistence theorem
       where each $p, q+1$ is critical or subcritical.
       Especially, for some specific case,
       we determined the range of $\lambda$ where a non-trivial solution
       $(u, \phi)$ does exist or does not exist.
 \item {\bf Generalized Joseph–-Lundgren exponent~[1]} \\
       This is a joint work with Prof.~Yasuhito Miyamoto (UTokyo).
       We worked on the following ordinary differential equation for $r
       \in (0, \infty)$:
       $r^{-(\gamma-1)} (r^\alpha \lvert u' \rvert^{\beta -1 } u')'
       + \lvert u \rvert^{p-1} u = 0$.
       Here the left term represents
       a generalized radial differential operator
       that covers, for example, the $N$-dimensional
       usual Laplacian, $m$-Laplacian or $k$-Hessian.
       In previous research
       the generalized Joseph–-Lundgren exponent for this operator
       was calculated, but there was a technical underbound for
       the exponent $p$.
       We removed that bound by transforming the equation
       and determined intersection numbers
       which role differently on $p$.
 \item {\bf Nonhomogeneous semilinear elliptic equations involving
       critical Sobolev exponent~[2]~[7]} \\
       I worked on the following
       nonhomogeneous semilinear elliptic equation
       involving the critical Sobolev exponent:
       $-\Delta u + a u = b u^p + \lambda f$.
       I proved that provided 
       $b$ achieves its maximum at an inner point of the
       domain and $a$ has a growth of the exponent $q$
       in some neighborhood of that point, then
       if the dimension of the domain is less than $6 + 2q$,
       there exist at least two positive solutions.
       It seems to be new that the coefficient of a linear term affects
       the dimension of the domain on which solutions exist.
 \item[] \underline{Mathematical informatics}
 \item {\bf Zero-dimensional fold and cut~[3]~[4]} \\
       This is a joint work with
       Yasuhiko Asao (UTokyo), Prof.~Erik D.~Demaine (MIT),
       Prof.~Martin L.~Demaine (MIT), Hideaki Hosaka (Azabu high school),
       Prof.~Akitoshi Kawamura (UTokyo)
       and Prof.~Tomohiro Tachi (UTokyo).
       We showed how to fold a piece of paper and punch one hole
       on given $n$ points
       so as to produce any desired patterns of holds.
       There is $4$ variants of problems;
       the paper is finite or infinite
       and we allow or forbid the crease on the points.
       In~[4], we gave solutions for each case and the order of crease
       are bounded on the polynomial order of $n$ and the paper ratio
       $r$.
       In the sequel paper [3], we also gave a definition of
       the complexity of folds, which
       will be useful for further studies that determine
       NP-hardness of complex folding problems.
 \item {\bf Application of SAT-solver for AI~[6]} \\
       It is known that $n$-satisfiability problems are NP-complete
       to solve for $n \geq 3$
       but are solved quickly by SAT-solver in recent years.
       I applied it for AI in the international
       programming contest ``SamurAI Coding
       2016--17'', which was held by Information
       Processing Society of Japan. I made an algorithm on SAT-solver
       to decide the all possible places of the hidden enemy logically
       by observing which places were conquested.
       It worked faster than a rudimentary algorithm by brute force.
 \item[] \underline{Social mathematics in FMSP}
 \item {\bf Control model for traffic lights} \\
       This is a joint work with Xinchi HUANG (UTokyo).
       We worked on discrete model of traffic lights which would not
       cause traffic jams.
       An observation data showed each number of cars for the pair of
       inlet and outlet of roads but there was ambiguity of
       the route of each cars.
       We let the problem come to
       $n$-varieties transportation problem but it is known as
       NP-complete. Therefore we also suggested an algorithm
       that superimpose usual max flow problems.
\end{enumerate}

\vspace{0.2cm}
\noindent
{\bf 発表論文}

\vspace{0.1cm}
%%%%%%%%%%%%%%%%%%%%%%%%%%%%%%%%%%%%%%%%%%%%%%%%%%%%%%%%%%%%%%%%%%%%%%%%%%%%%% 
% プレプリントも含めて,大学院進学後に発表したものをすべて書いて下さい。
%プレプリントarchiveに投稿したものは番号を記載して下さい。
% 様式は以下の例のように
% 著者・共著者名・ 題名・ジャーナル名・巻・年・ページの順に書いて下さい.
% タイトルの前に著者・共著者名を入れる形です。
% 共著の場合は著者名をすべて書いて下さい。
%%%%%%%%%%%%%%%%%%%%%%%%%%%%%%%%%%%%%%%%%%%%%%%%%%%%%%%%%%%%%%%%%%%%%%%%%%%%% 

\begin{enumerate}
 \item[] {\bf Refereed Papers}
 \item Yasuhito Miyamoto and Kazune Takahashi: ``Generalized
       Joseph-–Lundgren exponent and intersection properties for
       supercritical quasilinear elliptic equations'',
       Archiv der Mathematik {\bf 108} (2017) 71--83. 
 \item Kazune Takahashi: ``Semilinear elliptic equations with critical
       Sobolev exponent and non-homogeneous term'',
       Master Thesis, The University of Tokyo (2015).
 \item Yasuhiko Asao, Erik Demaine, Martin Demaine, Hideaki Hosaka,
       Akitoshi Kawamura, Tomohiro Tachi and Kazune Takahashi:
       ``Folding and Punching Paper'', to appear in
       Journal of Information Processing.
 \item[] {\bf Refereed Conference Abstracts}
 \item Yasuhiko Asao, Erik Demaine, Martin Demaine, Hideaki Hosaka,
       Akitoshi Kawamura, Tomohiro Tachi and Kazune Takahashi:
       ``Folding and Punching Paper'', Abstracts from the 19th Japan
       Conference on Discrete and Computational Geometry, Graphs and
       Games (2016) 40--41.
 \item[] {\bf Preprints}
 \item Hiroyuki Miyahara and Kazune Takahashi: ``Existence and
       Nonexistence of Standing Wave Solutions of 
       Nonlinear Schr\"{o}dinger-Poisson System'', preprint.
 \item[] {\bf Miscs}
 \item Kazune Takahashi: ``Application of SAT-solver for AI on SamurAI
       Coding 2016--17'', (2017),
       {\tt https:\slash\slash{}github.com\slash{}kazunetakahashi-thesis\slash{}SAT-solver-AI-project}.
 \item Kazune Takahashi: ``Semilinear elliptic equations with
       critical Sobolev exponent and non-homogeneous term'',
       to appear in RIMS K\^{o}ky\^{u}roku.
\end{enumerate}

\vspace{0.2cm}
\noindent
{\bf 口頭発表}

\vspace{0.1cm}
%%%%%%%%%%%%%%%%%%%%%%%%%%%%%%%%%%%%%%%%%%%%%%%%%%%%%%%%%%%%%%%%%%%%%%%%%%%%%%%
% 大学院進学後に行なった研究発表について
% 昨年度以前のものも含めて
% タイトル・シンポジウム(またはセミナー等)名・場所・月・年を 
% 書いて下さい.国際会議の場合は国名をお願いします.タイトルは原題で.
%%%%%%%%%%%%%%%%%%%%%%%%%%%%%%%%%%%%%%%%%%%%%%%%%%%%%%%%%%%%%%%%%%%%%%%%%%%%%%% 

\begin{enumerate}
 \item[] {\bf International Conferences}
 \item Semilinear elliptic equations with critical Sobolev exponent and
       non-homogeneous term, RIMS Workshop: Shapes and other properties
       of solutions of PDEs, RIMS, Kyoto University, Japan, Nov 2015.
       [Invited]
 \item (With Yasuhiko Asao, Erik Demaine, Martin Demaine, Hideaki
       Hosaka, Akitoshi Kawamura and Tomohiro Tachi)
       Folding and Punching Paper, The 19th Japan Conference on Discrete
       and Computational Geometry, Graphs, and Games, Tokyo University
       of Science, Japan, Sep 2016.
 \item[] {\bf Domestic Conferences}
 \item Existence and Nonexistence of Standing Wave Solutions of
       Nonlinear Schr\"{o}dinger-Poisson System,
       The 39th Differential Equation Seminar at Yokohama National
       University, Yokohama National University, Japan, Aug 2016.
       [Invited]
\end{enumerate}

\vspace{0.2cm}
\noindent
{\bf FMSPの活動への参加}

\vspace{0.1cm}
%%%%%%%%%%%%%%%%%%%%%%%%%%%%%%%%%%%%%%%%%%%%%%%%%%%%%%%%%%%%%%%%%%%%%%%%%%%%%% 
%本年度のFMSPが主催、共催する研究会、ワークショップ、FMSP Lectures、社会数理コロキウムなどへの参加を記入して下さい。また、それ以外に本年度FMSPから旅費等の補助を得て参加した国内外の研究会、セミナーなどの参加も記載して下さい。このような活動への参加によって何が得られたかを簡潔に書いて下さい。
%本年度スタディグループワークショップ、社会数理実践研究に参加した人
%はどのような課題を扱ったか、参加によって何が得られたか、自身がどのような貢献をしたかを書いて下さい。
%FMSPプログラムにおいて、長期海外渡航やインターンシップを行った
%場合は、2016年度だけではなく、すべて記載して下さい。また、それらについて、具体的にどのような研究活動を行ったか、どのような成果が得られたかをなるべく詳しく記述して下さい。
%%%%%%%%%%%%%%%%%%%%%%%%%%%%%%%%%%%%%%%%%%%%%%%%%%%%%%%%%%%%%%%%%%%%%%%%%%%%%%%

\begin{enumerate}
 \item[] {\bf Lectures}
 \item I attended ``Frontiers of Mathematical Sciences and Physics V (2014)'',
       ``VIII (2015)'', and ``VII (2016)'', and got grade A in all of
       them.
 \item[] {\bf Study Group}
 \item I attended ``Study Group of Environment Issues (Feb, 2016)''.
 \item[] {\bf Practical Studies of Mathematical Sciences and Society}
 \item Our result of this topic is summarized in
       ``6.~Control model for traffic lights'' above.
\end{enumerate}

\vspace{0.2cm}
\noindent
{\bf 受賞}

\vspace{0.1cm}
%%%%%%%%%%%%%%%%%%%%%%%%%%%%%%%%%%%%%%%%%%%%%%%%%%%%%%%%%%%%%%%%%%%%%%%%%% 
% 修士課程進学以降にありましたら書いて下さい. 研究科長賞などを含みます.
% 受賞年度を記入して下さい。
%%%%%%%%%%%%%%%%%%%%%%%%%%%%%%%%%%%%%%%%%%%%%%%%%%%%%%%%%%%%%%%%%%%%%%%%

\begin{enumerate}
 \item[] {\bf International Programming Contests}
 \item SamurAI Coding 2014--15, World Final: 6th place, 77th Information
       Processing Society of Japan National Convention, Kyoto
       University, Japan, Mar 2015.
 \item[] {\bf Domestic Programming Contests}
 \item Code Runner 2015, Final Round: 1st place,
       Recruit Career, Tokyo, Dec 2015.
 \item Code Runner 2014, Final Round: 7th place,
       Recruit Career, Tokyo, Nov 2014.
 \item Code Festival 2014 AI Challenge, Final Round: 3rd place,
       Recruit Holdings, Tokyo, Nov 2014.
\end{enumerate}

\vspace{0.4cm}

% Local Variables:
% mode: yatex
% coding: utf-8
% TeX-master: "FMSP_AnnualReport2016.tex"
% End: